\begin{titlepage}
	\begin{textblock*}{5cm}[0,0](15.1cm,1cm)
		\includegraphics[keepaspectratio,width=5cm]{img/HSLU_Logo}
	\end{textblock*}
	\begin{center}
		\vspace*{5cm}
		\Huge{\textbf{Machbarkeitsstudie}} \\
		\vspace{0.5em}
		\Large{RFID markierte Einzelexemplare}\\
		\vspace{3em}
		\LARGE{Pascal Baumann, Dane Wicki}\\
		\vspace{1em}
		\Large{Betreuer: Martin Jud}\\
		\vfill
		\large{Hochschule Luzern - Departement Informatik}\\
		\large{\today}\\
	\end{center}
	\begin{textblock*}{5cm}[0,0](15.3cm,277mm)
		\includegraphics[keepaspectratio,width=5cm]{img/FHZ_Logo}
	\end{textblock*}
\end{titlepage}

\renewcommand{\abstractname}{Management Summary}
\begin{abstract}
	In dieser Machbarkeitsstudie wird zu Beginn die Erkenntnisse aus dem gesamten Dokument aufgezeigt, mit der Empfehlung, ob das Projekt durchgeführt werden soll oder nicht. Anschliessend wird das Projekt beschrieben, wie dieses aufgebaut ist und wo die Kompetenzen liegen. Weiter werden die Beteiligten Parteien erwähnt sowie potenzielle Lieferanten für die RFID HF Hardwareausrüstung.
	Anschliessend folgt die Beschreibung des Umfeldes des Projektes, bei welchem näher auf das physische Umfeld der kooperativen Speicherbibliothek eingegangen wird, wie auch der Beschreibung des Problems.
	Darauf folgt eine Bestandsaufnahme, welche Konkurrenz, das mit diesem Projekt zu entwickelnde Produkt, bereits vorhanden ist sowie der Anschaffungsplan für die Materialien und eine kurze Analyse zur Verfügbarkeit der Arbeitskräfte.
	Weiter geht es mit den Technischen Charakteristiken, welche mit diesem Projekt umgesetzt werden soll, bei welchem genauer Beschrieben wird, welche Hardware zu verwenden ist und wie diese mit zu Platzieren ist. Zudem werden die Vorteile und deren Gründe aufgelistet für die Verwendung der beschriebenen Spezifikationen.
	In diesem Abschnitt findet sich auch die Detaillierte Auflistung, für die Kosten der Hardware, sowie die Kosten für weitere Anpassungen.
	Es folgt eine Auflistung der kritischen Punkte des Entwicklungsplanes, bei welchem jeder Punkt näher erläutert wird.
	Nach dem Entwicklungsplan wird auf die Kapitalvoraussetzungen und die Investition eingegangen und anschliessend auf die Betriebskosten.
	Beendet wird diese Machbarkeitsstudie mit der Analyse, ob das Projekt wirtschaftlich machbar ist.
	
\end{abstract}
