\chapter{Entwicklungs- und Produktionsplan}

\section{Kritische Punkte in der Entwicklung}
Wir sehen für die Entwicklung des Produktes vier kritische Punkte:
\begin{enumerate}
	\item Kickoff und Initialisierung
	\item Ausarbeitung Konzept
	\item Entwicklung und Implementierung
	\item Test und Abnahme
\end{enumerate}

\subsection{Initialisierung}
In dieser Phase soll die Planung erstellt werden, die Hardware beschafft und sich in das erarbeitete Konzept eingearbeitet werden. Weiter soll mit der Stöcklin durch die Teammitglieder oder den Kunden Kontakt aufgenommen und das Ausschleusen abgeklärt werden. Weiter soll in dieser Phase evaluiert werden, in welcher Programmiersprache die Lösung entwickelt wird.

\subsection{Ausarbeitung Konzept}
In dieser Phase sollen eventuelle Unklarheiten aus dem übergebenen Konzept, welche der erfolgreichen Durchführung im Weg stehen abgeklärt und eliminiert werden. Es soll eine Architektur entwickelt werden und eine CI/CD Pipeline eingerichtet werden.

\subsection{Entwicklung und Implementierung}
In dieser Phase soll die Software entwickelt werden und unter Laborbedingungen an der Hochschule selber getestet werden. Während dieser Phase wird von Stöcklin die Schnittstelle zur Ausschleusung eines Behälters zur Verfügung.

\subsection{Test und Abnahme}
In dieser letzten Phase soll die entwickelte Lösung beim Kunden installiert und getestet werden. Nachdem die Implementierung in das bestehende System der Stöcklin abgeschlossen ist, wird die Lösung durch den Kunden abgenommen.

\section{Kontrollprozesse für die Entwicklung}
Nach jeder abgeschlossenen Phase soll mit dem Kunden, der Speicherbibliothek, ein Meeting durchgeführt werden, bei dem die erarbeiteten Resultate präsentiert werden, das nächste Vorgehen besprochen und vom Kunden kontrollierend eingegriffen werden kann. Zur Kontrolle für beide beteiligten Parteien soll für jede Sitzung ein Protokoll angefertigt werden, welches bei der nächsten abgenommen wird.
