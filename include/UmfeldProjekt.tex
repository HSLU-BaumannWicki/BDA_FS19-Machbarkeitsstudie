\chapter{Allgemeines Umfeld und Bedürfnis für das Projekt}
In der Speicherbibliothek werden Exemplare der beteiligten Bibliotheken eingelagert. Das heisst, es wird gereinigt und inventarisiert eingelagert. Bibliotheksbenutzer können Exemplare im Bibliothekskatalog per Kurier in die jeweilige Bibliothek bestellen oder eine Digitale Kopie anfordern. Diese Digitale Kopie wird jedoch aus urheberrechtlichen Gründen wieder gelöscht, sollte also ein weiterer digitaler Auszug desselben Exemplar wieder angefordert werden, muss die gesamte Prozedur des Scanens wiederholt werden.

Sowohl die beteiligten Bibliotheken, wie auch die Bibliotheksbenutzer sind daher an einer zeitgerechten Lieferung interessiert. Dies bedeutet für die Speicherbibliothek, dass alle Prozesse und Abläufe effizient und zuverlässig ablaufen müssen.

\section{Physikalisches, Ökonomisches und Soziales Umfeld}
Die Speicherbibliothek befindet sich in Büron in Luzern. Sie ist gebaut auf dem Wies- und Ackerland der Grabmatte. Gesichert wird insbesondere das Hochlager durch Betonpfeiler, welche in den Boden getrieben sind, um das Gebäude zu stabilisieren. Der Boden ist flach aber Grundmass \parencite{MapGeoAdmin2019}.

Das Hochregallager hat die Dimensionen 70m x 20m x 15m, und, um die besten Lagerbedingungen für die Bücher herzustellen, eine Temperatur von 18\SIUnitSymbolDegree Celsius ($\pm$ 2\SIUnitSymbolDegree Veränderung während 48 Stunden) und eine relative Luftfeuchtigkeit von 45\% ($\pm$5\%). Als Brandschutzmassnahme wird der Sauerstoffgehalt im Lager von den normalen 21\% Luftsauerstoff durch Injektion von Stickstoff auf 13.5\% Reduziert. Dies verhindert die Entstehung von Bränden.
Die Bewirtschaftung des Lagers wird vollautomatisch durch Roboter bewerkstelligt.

Die Speicherbibliothek wurde durch eine eigens dafür gegründete Aktiengesellschaft aufgebaut. Der Kanton Luzern stellte dafür ein Grundstück in Büron als Sacheinlage plus eine separate Einlage (insgesamt vier Millionen Franken) in die Aktiengesellschaft und 24.8 Millionen Franken Beiträge in den Verein, welcher für den Betrieb zuständig ist. Der Kantonsrat stimmte dieser Vorlage im Februar 2013 zu. Das Stimmvolk Luzern stimmte im September 2013 über den Sonderkredit ab  \parencite{KantonLuzern2013}.

Der Betrieb der Bibliothek wird durch den Verein Kooperative Speicherbibliothek Schweiz getragen, welchem alle beteiligten Bibliotheken als Mitglieder beitreten und nach Platznutzung Beträge zahlen. Der Verein mietet das Gebäude von der Aktiengesellschaft und bietet die Dienstleistungen wie Bestellungsabwicklung und Rückgaben für die Mitglieder an. Beide, der Verein wie auch die Aktiengesellschaft, sind nicht gewinnorientierte Unternehmen.

\section{Regionale, Nationale und Internationale Relevanz zum Projekt}
Es existieren keine Fertiglösungen oder Referenzprojekte welche die gleiche Problemstellung bewältigen. Daher besitzt dieses Projekt ein gewisses Neuerungspotential. Die Effekte sind jedoch lokal beschränkt und haben daher nur geringe Relevanz.

\section{Relevante Regierungsvorschriften und Anreize}
Da RFID im Funkspektrum agiert, bestehen regulatorische Vorschriften bezüglich der Sendestärke. So darf in der Schweiz auf dem 13.56MHz Frequenzband für Nahbereichsanwendung die Sendestärke auf zehn Metern maximal 60 dB$\mu$A betragen.

Auf der Seite der Applikationsentwicklung schreibt der Gesetzgeber keine Massnahmen vor. Der Datenschutz greift erst, wenn personenbezogene Daten betroffen sind. Da im Projekt mit Inventardaten gearbeitet wird, greift dieser nicht.

Nach eigener Recherche konnten keine relevanten finanzielle Anreize für unser Projekt weder auf kantonaler Ebene noch auf Bundesebene identifiziert werden.

\section{Beschreibung des Problems}
In der Speicherbibliothek in Büron, Luzern ist ein Behälter-Hochregallager erbaut worden. In diesem Hochregallager werden in bis zu 110'000 Behältern verschiedene Exemplare (welche viele mit RFID Tags ausgestattet sind) gelagert. Die Behälter werden manuell von Menschen befüllt und anschliessend wird der Behälter voll autonom an einen Lagerplatz gefahren. Zeitweise können auch gewisse Exemplare wieder aus den Behältern entnommen werden um diese zu Lesen, Scannen oder einer der teilnehmenden Bibliotheken zurückzusenden. Während dem Vorgang des Lagerns und Entnehmens der Exemplare werden weiterhin Menschen für das Befüllen/Entleeren der Behälter verwendet. Dies birgt die Gefahr, dass eine Person aus Versehen ein Exemplar in einen falschen Behälter legt. Und so das Exemplar nur sehr umständlich wiedergefunden werden kann \parencite{HochschuleLuzern2019}.

Um diesen Umstand zu Verhindern oder den Prozess des Wiederfindens zu beschleunigen, soll in diesem Projekt abgeklärt werden, was für Lösungsansätze existieren und wie machbar diese sind.

\section{Effekt auf betroffene Parteien}
Die Mitglieder der Bibliotheken können weiterhin den hohen Qualitätsstandard der Speicherbibliothek geniessen.
Die Mitarbeiter können sich auf ihre Kernarbeiten konzentrieren.

\section{Verhinderer des Projektes}
Die Entwicklerfirma des Lagerverwaltungssystems könnte unter Umständen die Einbindung der Scanstation ablehnen. Da die Bindung zu der Entwicklerfirma jedoch sehr stak ist, kann ein Wechsel dieser nur schwer umgesetzt werden.
