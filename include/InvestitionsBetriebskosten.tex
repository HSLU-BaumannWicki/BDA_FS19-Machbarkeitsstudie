\chapter{Kapitalvoraussetzungen und Investitionsplan}
Für die Umsetzung dieses Projektes muss in folgendes Investiert werden:

\vspace{1em}

\begin{tabularx}{\textwidth}{|X|r|}
	\hline
	\textbf{Beschreibung} & \textbf{Kosten (CHF)} \\
	\hline
	RFID Reader ID ISC.LR2500-A (Feig) & 1'300 \\
	\hline
	RFID Antennen ID ISC.ANT800/600 (Feig)& 1'660 \\
	\hline
	RFID Antennenkabel ID ISC.ANT.C-A (Feig) & 44 \\
	\hline
	RFID Multiplexer 8-fach HF Multiplexer (Feig) & 540 \\
	\hline
	Netzteil ID NET.24V-B (Feig) & 33 \\
	\hline
	Netzkabel ID CAB.NET.24V-B-EU (Feig) & 5 \\
	\hline
	Montagehalterung & 250 \\
	\hline
	Raspberry Pi 3 Model B+ (pi-shop.ch) & 39 \\
	\hline
	Sandisk Extreme 128GB Class 10 (digitec.ch) & 45 \\
	\hline
	Raspberry Pi Netzteil (digitec.ch) & 25 \\
	\hline
	Softwareanpassungen Stöcklin & 10'000 \\
	\hline
	Durchführung des Projektes & 1'000 \\
	\hline
	\hline
	 & \textbf{14'941} \\
	 \hline
\end{tabularx}

\vspace{1em}

Um das Projekt durchführen zu können wird demnach ein mindestkapital von 14'941 Franken vorausgesetzt.

\chapter{Geschätzte Betriebskosten}

\chapter{Wirtschaftliche Machbarkeit des Projekts}

\chapter{Finanzieller Plan des Projekts}
