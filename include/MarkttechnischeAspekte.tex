\chapter{Marktpotential für Güter und Dienste welche im Laufe des Projekts entwickelt werden}

Gemäss unserer Recherchen gibt es noch keine Lösung auf dem Markt, die RFID markierte Exemplare automatisch den zugehörigen Behältern zuweist und ein Abgleich mit dem Lagerverwaltungssystem vornimmt. High Frequency RFID Tags werden vor allem im Bibliotheksumfeld eingesetzt, potentielle Kunden wären daher Bibliotheken und deren Aussenlager die mit ähnlichen Systemen arbeiten. \citeauthor{Niederer2017} identifiziert nur die British Library mit ihren 'additional storage buildings', welche mit einem sauerstoffarmen, vollautomatisierten Lagersystem ausgerüstet ist, identifiziert das System jedoch als Zukunftsweisend. Es ist also anzunehmen, das die Anzahl solcher Systeme, und somit die Anzahl potentieller Interessenten, zunehmend ist.

\begin{table}[h!]
	\centering
	\begin{tabularx}{\textwidth}{|X|X|}
		\hline
		\textbf{Strengths} & \textbf{Weaknesses} \\
		\hline
		\begin{itemize}[noitemsep]
			\item Automatisierte Auffindung
			\item Automatisiertes Ausschleusen
		\end{itemize} & \begin{itemize}[noitemsep]
			\item Nicht hundertprozentige Auffindungsrate
		\end{itemize}
		\\
		\hline
		\textbf{Opportunities} & \textbf{Threats} \\
		\hline
		\begin{itemize}[noitemsep]
			\item Neue Lösung
			\item Keine Konkurrenz
		\end{itemize} & \begin{itemize}[noitemsep]
			\item Neuheit der Lösung, somit anzunehmende Fehler
		\end{itemize}
		\\
		\hline
	\end{tabularx}
\end{table}

\chapter{Verfügbarkeit der Arbeitskraft}
Das Gerät wird in einer weiteren Bachelorarbeit entwickelt. Die Verfügbarkeit von Studenten wird als unkritisch angesehen, jedoch muss die ausgeschriebene Arbeit auch von Studenten ausgewählt und angenommen werden. Da das Projekt auch die Halterungen der Antenne auch in der Arbeit erstellt werden müssen, ist es vorstellbar dies in einer Interdisziplinären Arbeit zu erstellen. Dies würde aber dementsprechend die Komplexität erhöhen, in Anbetracht dessen, dass nun mehrere Abteilungen oder sogar Hochschulen am gleichen Projekt beteiligt wären.

Das Risiko bei Studenten als Arbeitskräften liegt in derer potentiellen Unerfahrenheit in der Durchführung und Erarbeitung solcher Projekte.
