\chapter{Allgemeines Umfeld und Bedürfnis für das Projekt}
In der kooperativen Speicherbibliothek werden Exemplare der beteiligten Bibliotheken dedoubliert eingelagert. Das heisst es wird jeweils das besterhaltene Exemplar ausgewählt, gereinigt und inventarisiert eingelagert. Sollte ein Bibliotheksbenutzer ein Exemplar dieses Werks anfordern, wird das Exemplar entweder per Kurier an die jeweilige Bibliothek verschickt, oder bei Journalen oder Magazinen vorzugsweise, eingescannt und in digitaler Form an den Benutzer übergeben. Diese digitalen Kopien werden nach Versand an den Kunden aus urheberrechtlichen Gründen wieder gelöscht, sollte also ein weiterer digitaler Auszug desselben Exemplar angefordert werden, muss das ganze Prozedere wiederholt werden.

Sowohl die beteiligten Bibliotheken wie auch die Bibliotheksbenutzer sind daher an einer zeitgerechten Lieferung interessiert. Dies bedeutet für die Speicherbibliothek, dass alle Prozesse und Abläufe effizient und zuverlässig ablaufen müssen. Da Verzögerungen in Zwischenschritten sich auf die ganze Auslieferung des Exemplars, oder der digitalen Kopie desjenigen, auswirken.

\section{Physikalisches, Ökonomisches und Soziales Umfeld}
% Physikalisch
Die Kooperative Speicherbibliothek befindet sich in Büron in Luzern.
% Land, Sumpfig, Normal, Wetter
% Bedingungen im Lager

% Volksentscheid
% Abstimmungsresultat
% Abstimmungsresultat in Büron
% Vereinsstruktur
% Kredit über 10 Jahre?

\section{Regionale, Nationale und Internationale Relevanz zum Projekt}

\section{Relevante Regierungsvorschriften und Anreize}

\section{Beschreibung des Problems}

\section{Effekt auf betroffene Parteien}

\section{Verhinderer des Projektes}
