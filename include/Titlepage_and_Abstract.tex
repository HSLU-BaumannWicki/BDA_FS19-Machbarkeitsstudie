\begin{titlepage}
	\begin{textblock*}{5cm}[0,0](15.1cm,1cm)
		\includegraphics[keepaspectratio,width=5cm]{img/HSLU_Logo}
	\end{textblock*}
	\begin{center}
		\vspace*{5cm}
		\Huge{\textbf{Machbarkeitsstudie}} \\
		\vspace{0.5em}
		\Large{RFID markierte Einzelexemplare}\\
		\vspace{3em}
		\LARGE{Pascal Baumann, Dane Wicki}\\
		\vspace{1em}
		\Large{Betreuer: Martin Jud}\\
		\vfill
		\large{Hochschule Luzern - Departement Informatik}\\
		\large{\today}\\
	\end{center}
	\begin{textblock*}{5cm}[0,0](15.3cm,277mm)
		\includegraphics[keepaspectratio,width=5cm]{img/FHZ_Logo}
	\end{textblock*}
\end{titlepage}

\renewcommand{\abstractname}{Inhalt des Dokument}
\begin{abstract}
	In dieser Machbarkeitsstudie werden zu Beginn die Erkenntnisse aus dem gesamten Dokument aufgezeigt, mit der Empfehlung, dass das Projekt durchgeführt werden soll. Anschliessend wird beschrieben, wie das Projekt aufgebaut ist und wo die Kompetenzen liegen. Weiter werden die Beteiligten Parteien erwähnt sowie potenzielle Lieferanten für die RFID HF Hardwareausrüstung identifiziert.
	Anschliessend folgt die Beschreibung des Umfeldes des Projektes, bei welchem näher auf das physische Umfeld der kooperativen Speicherbibliothek eingegangen wird, wie auch auf eine genaue Beschreibung des Problems.
	Darauf folgt eine Bestandsaufnahme, welche Konkurrenz, des mit diesem Projekt zu entwickelnde Produkt, bereits vorhanden ist, sowie der Anschaffungsplan für die Materialien und eine kurze Analyse zur Verfügbarkeit der Arbeitskräfte.
	Weiter werden die technischen Charakteristiken analysiert, welche mit diesem Projekt umgesetzt werden sollen, bei welchen genauer beschrieben wird, welche Hardware zu verwenden ist und wie diese zu positionieren ist. Zudem werden die Vorteile und  Gründe aufgelistet für die Verwendung der beschriebenen Teile.
	In diesem Abschnitt findet sich auch die detaillierte Auflistung für die Kosten der Hardware, sowie der Kosten für weitere Anpassungen.
	Es folgt eine Auflistung der kritischen Punkte des Entwicklungsplanes, bei welchem jeder Punkt näher erläutert wird.
	Nach dem Entwicklungsplan wird auf die Kapitalvoraussetzungen und die Investition eingegangen und anschliessend auf die Betriebskosten.
	Zum Schluss wird analysiert, ob das Projekt wirtschaftlich machbar ist.
\end{abstract}

\renewcommand{\abstractname}{Management Summary}
\begin{abstract}
	Diese Machbarkeitsstudie wurde im Rahmen einer Bachelorarbeit der Hochschule Luzern für die Kooperative Speicherbibliothek Schweiz entwickelt. Dabei wurden für die Lösung eines Problems zwei unterschiedliche Konzepte entwickelt, von welchen das technisch Realisierbare für diese Machbarkeitsstudie selektiert wurde. Dieses Konzept löst das Problem, dass ein Exemplar, welche in Behältern im Hochregallager eingelagert werden, in einen falschen Behälter einsortiert wird.
	
	In dieser Machbarkeitsstudie wurden die Entscheidungen, welche im Konzept getroffen wurden, auf deren Wirtschaftlichkeit sowie der technologischen Umsetzungsmöglichkeit überprüft. 
	So wurde festgehalten, dass es technisch nicht möglich ist 100\% aller mit RFID-Tags ausgerüsteten Exemplare zu erkennen. Dies ist der Fall, sobald die Tags zu nahe aufeinander liegen.
		
	Bei der wirtschaftlichen Überprüfung konnte festgestellt werden, dass es keinen Gesamtlösungsanbieter gibt, welcher dieses Problem bei dieser Tag-dichte lösen kann. Es konnte zudem ermittelt werden, dass die Umsetzung des Konzeptes in einem Projekt circa 20'000 Franken kostet. Das Wirtschaftlich grösste Risiko konnte bei Stöcklin identifiziert werden, da diese eine unabhängige, weitere Partei darstellt, welche das momentan verwendete Lagerverwaltungssystem liefert. Diese Machbarkeitsstudie kommt zum Ergebnis, dass eine Umsetzung das Risiko genug stark minimieren kann, sodass mit der Umsetzung der potenzielle wirtschaftliche Schaden die Umsetzungskosten von ca. 20'000 Franken eingespart werden können.
\end{abstract}
