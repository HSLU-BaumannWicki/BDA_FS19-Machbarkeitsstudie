\chapter{Beschreibung des Projektes}

\section{Natur des Projektes}
Im Rahmen des Projektes soll ein neuer automatischer Arbeitsprozess erstellt werden, welcher sicherstellt, dass mit RFID ausgerüstete Exemplare nicht in einen falschen Behälter platziert werden können. Dieser Arbeitsprozess soll sich in die bestehenden Arbeitsprozesse zur Einlagerung von Exemplaren im Hochregallager der Speicherbibliothek in Büron vollständig integrieren. Dazu soll eine Applikation erstellt werden, welche im Zusammenspiel mit RFID Lesehardware sowie der im Einsatz befindlichen Lagerverwaltungssoftware automatisch bei einem deplatzierten Exemplar den Behälter ausschleust. Das Projekt soll im Rahmen einer Bachelorarbeit entwickelt werden.

\section{Beteiligte Parteien im Umfeld des Projektes}
* Hochschule
* Speicherbibliothek
* Lagerverwaltungssoftware
* bestehende Prozesse
* Gesetzgeber

\section{Struktur des Projektes}
Das Projekt wird durch Studenten des Fachbereiches Informatik in aktiver Zusammenarbeit der Kooperativen Speicherbibliothek durchgeführt. Währen dieser Durchführung liegt die Entscheidungsgewalt bei den Studenten. Mit Fertigstellung des Projektes wird das Produkt der Speicherbibliothek übergeben. Diese übernimmt anschliessend die Entscheidungsgewalt wie auch die Verantwortung der Wartung sowie Weiterentwicklung.

\section{Zielgruppe und Konkurrenten}
Die Zielgruppe ist jede Unternehmung, welches sich zum Ziel gesetzt hat, mehrere mit RFID ausgestattete Objekte indexiert in einem Behältnis zu lagern und das manuelle Deplatzieren eines Objektes in ein falsches Behältnis zu unterbinden.
Dazu gehörten Unternehmen mit einem Hochregallager, in welchem sie Exemplare wie zum Beispiel Bücher oder Ordner in Behältnissen lagern und dabei erfassen, welche Exemplare in welchen Behältern gelagert ist.

Es bestehen momentan keine direkten Konkurrenzprodukte, welche verschiedene Exemplare in einem Behältnis auffindet. Ansatzweise existieren jedoch bereits Produkte, welche ihrerseits Behälter oder Paletten mittels RFID identifizieren und so in einem Lager ein automatisches Checkout/in umsetzen können. Jedoch sind diese Produkte noch nicht so ausgelegt, dass diese viele kleinere Objekte innerhalb eines solchen Behältnisses identifizieren können.

\section{Lieferanten}

\section{Material- und Mitarbeiterkosten}
