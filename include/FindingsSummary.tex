\chapter{Zusammenfassung der wichtigsten Erkenntnisse}
Bedingt durch die niedrigen Initialkostenaufwand, sowie den eher geringen Betriebskosten würde eine Umsetzung dieses Projektes einen Mehrwert bringen.
Durch die Umsetzung des Projektes kann verhindert werden, dass Kosten für eine manuelle Suche anfallen, sowie das ein Reputationsschaden entsteht, welcher bis zum Abgang einer Bibliothek führen kann.

\vspace{2em}

\noindent

Technisch ist das Projekt mit folgenden Limitationen bezüglich der Lesbarkeit der Tags realisierbar:
\begin{itemize}
	\item Nicht lesbar bei dünnen Exemplaren (Dicke < 3cm)
	\item Keine Unterstützung von aufeinandergestapelten Exemplaren
\end{itemize}

\vspace{2em}

\noindent
Die Erkenntnisse dieser Machbarkeitsstudie führen zum Schluss, dass dieses Projekt trotz der technischen Limitationen durchzuführen ist.
Weiter ist es wichtig, dass bereits früh die RFID HF Hardware von Feig Electronic zu bestellt, sowie dass bei Beginn der Arbeit die Zusammenarbeit mit Stöcklin bezüglich des eingesetzten Lagerverwaltungssystems gesucht wird.
