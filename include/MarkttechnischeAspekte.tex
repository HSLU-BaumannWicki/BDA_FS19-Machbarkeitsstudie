\chapter{Marktpotential für Güter und Dienste welche im Laufe des Projekts entwickelt werden}

Gemäss unserer Recherchen gibt es noch keine Lösung auf dem Markt, die RFID markierte Exemplare automatisch den zugehörigen Behältern zuweist und ein Abgleich mit dem Lagerverwaltungssystem vornimmt. High Frequency RFID Tags werden vor allem im Bibliotheksumfeld eingesetzt, potentielle Kunden wären daher Bibliotheken und deren Aussenlager die mit ähnlichen Systemen arbeiten. \citeauthor{Niederer2017} identifiziert nur die British Library mit ihren 'additional storage buildings', welche mit einem sauerstoffarmen, vollautomatisierten Lagersystem ausgerüstet ist, identifiziert das System jedoch als Zukunftsweisend. Es ist also anzunehmen, das die Anzahl solcher Systeme, und somit die Anzahl potentieller Interessenten, zunehmend ist.

\begin{table}[h!]
	\centering
	\begin{tabularx}{\textwidth}{|X|X|}
		\hline
		\textbf{Strengths} & \textbf{Weaknesses} \\
		\hline
		\begin{itemize}[noitemsep]
			\item Automatisierte Auffindung
			\item Automatisiertes Ausschleusen
		\end{itemize} & \begin{itemize}[noitemsep]
			\item Nicht hundertprozentige Auffindungsrate
		\end{itemize}
		\\
		\hline
		\textbf{Opportunities} & \textbf{Threats} \\
		\hline
		\begin{itemize}[noitemsep]
			\item Neue Lösung
			\item Keine Konkurrenz
		\end{itemize} & \begin{itemize}[noitemsep]
			\item Neuheit der Lösung, somit anzunehmende Fehler
		\end{itemize}
		\\
		\hline
	\end{tabularx}
\end{table}

\chapter{Materialvoraussetzung und Anschaffungsplan}
\label{sec:Materialvoraussetzung_Anschaffungsplan}
Da die Distanz bis zum Boden oder Seite eines Behälters nie mehr als 50cm beträgt, ist eine Lesedistanz von 60cm oder höher ausreichend. In den Versuchen wurde herausgefunden, dass die Tags einen Mindestabstand von 3cm zueinander besitzen müssen, ansonsten schlägt das Auslesen fehl.

Weiter wurde herausgefunden, dass sowohl Stahlblech und Aluminium das Auslesen beeinträchtigen oder verhindern, jedoch nur wenn sie sich zwischen Leseantenne und Tag befinden. Eine Halterung kann also gut aus diesen Materialien gefertigt werden. Aus Kostengründen wird Aluminium empfohlen. Die Halterungen müssen das Gewicht (1.9kg) der Antennen halten. Überschlagsrechnungen haben ergeben, dass dafür ein Profil aus Aluminium gebraucht werden kann.

\section{Kosten und Anschaffungsplan}
Bei der RFID Lösung wurde sich für Feig Electronics entschieden, da diese sehr gute Hardware zu einem vernünftigen Preis liefern, wie auch, dass direkt auf dem Leser ein embedded Linux zur Verfügung gestellt wird, wodurch die externe Steuerung des Lesers wegfällt.

Auf Rat eines Experten, der item Industrietechnik empfahl, wurde sich entschieden diese für die Lieferung der Montagehalterung auszuwählen, da diese Fertigprofile mit den gewünschten Materialeigenschaften liefern.

\begin{table}[h!]
	\centering
	\begin{tabularx}{\textwidth}{|X|X|X|}
		\hline
		\textbf{Material} & \textbf{Lieferant} & \textbf{Lieferzeit} \\
		\hline
		RFID Reader und Zubehör & Feig Electronics & 2 Wochen\\
		\hline
		Montagehalterung & item Industrietechnik GmbH & 2 Wochen\\
		\hline
	\end{tabularx}
\end{table}

\chapter{Verfügbarkeit der Arbeitskraft}
Das Gerät wird in einer weiteren Bachelor- oder Wirtschaftsarbeit entwickelt. Die Verfügbarkeit von Studenten wird als unkritisch angesehen, jedoch muss die ausgeschriebene Arbeit auch von Studenten ausgewählt und angenommen werden. Da im Projekt auch die Halterungen der Antenne erstellt werden müssen, ist es vorstellbar das Projekt in einer Interdisziplinären Form aufzuziehen. Dies würde aber dementsprechend die Komplexität erhöhen, in Anbetracht dessen, dass nun mehrere Abteilungen oder sogar Hochschulen am gleichen Projekt beteiligt wären.

Das Risiko bei Studenten als Arbeitskräften liegt in derer potentiellen Unerfahrenheit in der Durchführung und Erarbeitung solcher Projekte.
