\chapter{Allgemeines Umfeld und Bedürfnis für das Projekt}
In der kooperativen Speicherbibliothek werden Exemplare der beteiligten Bibliotheken  eingelagert. Das heisst es wird jeweils das besterhaltene Exemplar ausgewählt, gereinigt und inventarisiert eingelagert. Sollte ein Bibliotheksbenutzer ein Exemplar dieses Werks anfordern, wird das Exemplar entweder per Kurier an die jeweilige Bibliothek verschickt, oder bei Journalen oder Magazinen vorzugsweise, eingescannt und in digitaler Form an den Benutzer übergeben. Diese digitalen Kopien werden nach Versand an den Kunden aus urheberrechtlichen Gründen wieder gelöscht, sollte also ein weiterer digitaler Auszug desselben Exemplar angefordert werden, muss das ganze Prozedere wiederholt werden.

Sowohl die beteiligten Bibliotheken, wie auch die Bibliotheksbenutzer sind daher an einer zeitgerechten Lieferung interessiert. Dies bedeutet für die Speicherbibliothek, dass alle Prozesse und Abläufe effizient und zuverlässig ablaufen müssen. Da Verzögerungen in Zwischenschritten sich auf die ganze Auslieferung des Exemplars oder der digitalen Kopie desjenigen, auswirken.

\section{Physikalisches, Ökonomisches und Soziales Umfeld}
Die Kooperative Speicherbibliothek befindet sich in Büron in Luzern. Sie ist gebaut auf dem Wies- und Ackerland der Grabmatte. Gesichert wird insbesondere das Hochlager durch Betonpfeiler, welche in den Boden getrieben sind, um das Gebäude zu stabilisieren. Der Boden ist flach aber grundnass \parencite{GeoMapAdmin2019}.

Das Hochregallager hat die Dimensionen 70m x 20m x 15m, und um die besten Lagerbedingungen für die Bücher herzustellen, eine Temperatur von 18\SIUnitSymbolDegree ($\pm$ 2) und eine relative Luftfeuchtigkeit von 45\% ($\pm$5\%). Als Brandschutzmassnahme wird der Sauerstoffgehalt im Lager von den normalen 21\% Luftsauerstoff durch Injektion von Stickstoff auf 13.5\% gesenkt. Dies unterdrückt die Ausbreitung von Bränden.
Die Bewirtschaftung des Lagers wird vollautomatisch durch Roboter bewerkstelligt.

Die Speicherbibliothek wurde durch eine eigens dafür gegründete Aktiengesellschaft getragen. Der Kanton Luzern stellte dafür ein Grundstück in Büron als Sacheinlage, einen Anteil von vier Millionen Franken für Einlagen in die Aktiengesellschaft und 24.8 Millionen Franken Beiträge in den Verein, welcher für den Betrieb zuständig ist. Der Kantonsrat stimmte dieser Vorlage im Februar 2013 zu. Das Stimmvolk Luzern stimmte im September 2013 über den Sonderkredit ab. 

Der Betrieb der Bibliothek wird durch den Verein Kooperative Speicherbibliothek Schweiz getragen, welchem alle beteiligten Bibliotheken als Mitglieder beitreten und nach Platznutzung Beträge zahlen. Der Verein mietet das Gebäude von der Aktiengesellschaft und bietet die Dienstleistungen wie Bestellungsabwicklung und Rückgaben für die Mitglieder an.

\section{Regionale, Nationale und Internationale Relevanz zum Projekt}
Es existieren keine Fertiglösungen oder Referenzprojekte, welche die gleiche Problemstellung bewältigen. Es besitzt daher gewisses Neuerungspotential. Die Effekte sind jedoch lokal beschränkt und haben daher nur geringe Relevanz.

\section{Relevante Regierungsvorschriften und Anreize}
Da RFID im Funkspektrum agiert, bestehen regulatorische Vorschriften bezüglich der Sendestärke. So darf in der Schweiz auf dem 13.56MHz Frequenzband für Nahbereichsanwendung die Sendestärke auf zehn Metern maximal 60 dB$\mu$A betragen.

Auf der Seite der Applikationsentwicklung schreibt der Gesetzgeber keine Massnahmen vor. Der Datenschutz greift erst, wenn personenbezogene Daten betroffen sind. Da im Projekt mit Inventardaten gearbeitet wird, greift dieser nicht.

Es existieren keine relevanten finanziellen Anreize für unser Projekt vonseiten der Regierung.

\section{Beschreibung des Problems}
In der Speicherbibliothek in Büron, Luzern ist ein Behälter-Hochregallager erbaut worden. In diesem Hochregallager werden bis zu 110'000 Behältern verschiedene Exemplare (welche viele mit RFID Tags ausgestattet sind) gelagert. Die Behälter werden manuell von Menschen befüllt und anschliessend wird der Behälter voll autonom an einen Lagerplatz gefahren. Zeitweise können auch gewisse Exemplare wieder aus den Behältern entnommen werden um dies zu Lesen, Scannen oder einer der teilnehmenden Bibliotheken zurückzusenden. Während dem Vorgang des Lagerns und Entnehmen der Exemplare werden weiterhin Menschen für das Befüllen/Entfernen der Behälter verwendet. Dies birgt die Gefahr, dass eine Person aus Versehen ein Exemplar in einen falschen Behälter legt. Und so das Exemplar nur sehr umständlich wiedergefunden werden kann.
\parencite(HochschuleLuzern2019)

Um diesen Umstand zu Verhindern oder den Prozess des Wiederfindens zu beschleunigen, soll in diesem Projekt abgeklärt werden, was für Lösungsansätze existieren und wie machbar diese sind.

\section{Effekt auf betroffene Parteien}
Die Mitglieder der Bibliotheken können weiterhin den hohen Arbeitsstandard der Speicherbibliothek geniessen.
Die Mitarbeiter können sich auf ihre Kernarbeiten konzentrieren.

\section{Verhinderer des Projektes}
Die Entwicklerfirma des Lagerverwaltungssystems könnte unter Umständen die Einbindung der Scanstation ablehnen und da die Bindung zu dieser stark ist, kann ein Wechsel nur schwer gemacht werden.
